\documentclass[10pt,landscape]{article}
\usepackage{multicol}
\usepackage{calc}
\usepackage{ifthen}
\usepackage[landscape]{geometry}
\usepackage{amsmath,amsthm,amsfonts,amssymb}
\usepackage{color,graphicx,overpic}
\usepackage{hyperref}

% This sets page margins to .5 inch if using letter paper, and to 1cm
% if using A4 paper. (This probably isn't strictly necessary.)
% If using another size paper, use default 1cm margins.
\ifthenelse{\lengthtest { \paperwidth = 11in}}
    { \geometry{top=.5in,left=.5in,right=.5in,bottom=.5in} }
    {\ifthenelse{ \lengthtest{ \paperwidth = 297mm}}
        {\geometry{top=1cm,left=1cm,right=1cm,bottom=1cm} }
        {\geometry{top=1cm,left=1cm,right=1cm,bottom=1cm} }
    }

% Turn off header and footer
\pagestyle{empty}

% Redefine section commands to use less space
\makeatletter
\renewcommand{\section}{\@startsection{section}{1}{0mm}%
                                {-1ex plus -.5ex minus -.2ex}%
                                {0.5ex plus .2ex}%x
                                {\normalfont\large\bfseries}}
\renewcommand{\subsection}{\@startsection{subsection}{2}{0mm}%
                                {-1explus -.5ex minus -.2ex}%
                                {0.5ex plus .2ex}%
                                {\normalfont\normalsize\bfseries}}
\renewcommand{\subsubsection}{\@startsection{subsubsection}{3}{0mm}%
                                {-1ex plus -.5ex minus -.2ex}%
                                {1ex plus .2ex}%
                                {\normalfont\small\bfseries}}
\makeatother

% Define BibTeX command
\def\BibTeX{{\rm B\kern-.05em{\sc i\kern-.025em b}\kern-.08em
    T\kern-.1667em\lower.7ex\hbox{E}\kern-.125emX}}

% Don't print section numbers
\setcounter{secnumdepth}{0}

\setlength{\parindent}{0pt}
\setlength{\parskip}{0pt plus 0.5ex}

% Environments
\newtheorem{example}[section]{Example}
% -----------------------------------------------------------------------

\begin{document}
\raggedright
\footnotesize
\begin{multicols}{3}


% multicol parameters
% These lengths are set only within the two main columns
%\setlength{\columnseprule}{0.25pt}
\setlength{\premulticols}{1pt}
\setlength{\postmulticols}{1pt}
\setlength{\multicolsep}{1pt}
\setlength{\columnsep}{2pt}

\begin{center}
     \Large{FPM Algebra } \\
     \footnotesize{Marie Biolkov\'a}
\end{center}

\section{Functions}
Proving functions: if $x=y$ then $f(x) =f(y)$. \\
A function $f : X \rightarrow Y$ is called
\begin{itemize}
	\item \emph{injective} if $f(x_1) = f(x_2)$ implies that $x_1 = x_2$.
	\item \emph{surjective} if for every $y \in Y$, there exists $x \in X$ such that $f(x) = y$.
	\item \emph{bijective} if it is both injective and surjective.
\end{itemize}

%\section{Graphs, Symmetries and Groups}
%\textbf{Definition 1.1.1} A \emph{graph} is a finite set of vertices joined by edges. We will assume that there is at most one edge joining two given vertices and no edge joins a vertex to itself. The \emph{valency} of a vertex is the number of edges emerging from it.

%\textbf{Definition 1.1.3} An \emph{isomorphism} between two graphs is a bijection between them that preserves all edges. More precisely, if $ \Gamma_{1} $ and $ \Gamma_{2} $ are graphs, with sets of vertices $V_1$ and $V_2$, respectively, then an isomorphism from $ \Gamma_{1} $ to $\Gamma_{2} $ is a bijection $f : V_1 \rightarrow V_2$ such that $f(v_1)$ and $f(v_2)$ are joined by an edge if and only if $v_1$ and $v_2 $ are joined by an edge.
%We say that $ \Gamma_{1} $ and $\Gamma_{2} $ are \emph{isomorphic} if there exists an isomorphism $f: \Gamma_{1} \rightarrow \Gamma_{2}$.

%\textbf{Definition 1.1.9} A \emph{symmetry} of a graph is an isomorphism of a graph to itself, i.e. if the set of vertices is $V$ , then a symmetry is a bijection $f: V \rightarrow V$ that preserves edges. That is, a symmetry is a bijection $f: V \rightarrow V$ such that $f(v_1)$ and $f(v_2)$ are joined by an edge if and only if $f(v_1)$ and $f(v_2)$ are joined by an edge.

\section{Group Axioms} 
We say that a nonempty set $G$ is group under * if 
	\begin{enumerate}
		\item (Closure) * is an operation, so $ g * h \in G$ for all $g,h, \in G$.
		\item (Associativity) $g*(h*k) = (g*h)*k$ for all $g,h,k \in G$.
		\item (Identity) There exists an \emph{identity element} $e \in G$ such that $e*g = g*e = g$ for all $g \in G$.
		\item (Inverses) Every element $g\in G$ has an inverse $g^{-1}$ such that $g*g^{-1} = g^{-1}*g=e$.
\end{enumerate} 

\section{Subgroups}
A \emph{proper subgroup} is a subgroup that is not the group itself (sometimes denoted $ H < G$).
If $H \leq G$ then $e_H = e_G$ and the inverse of $h \in H $ equals the inverse of $h$ in $G$.

	\subsection{Test for a Subgroup} 
	We say $ H \subseteq G $ is a subgroup of $G$ if and only if
	\begin{enumerate}
		\item $H$ is not empty.
		\item If $h,k, \in H$ then $h * k \in H$.
		\item If $h \in H $ then $h^{-1} \in H$. 
	\end{enumerate} 
	Note: associativity is inherited from $G$.\\
	The union of subgroups is not a subgroup! The intersection is.

\section{Lagrange \& Co.}
\textbf{Lagrange's Theorem} Let $G$ be a finite group and let $H \leq G$. Then $|H|$ divides $|G|$.
\begin{itemize}
	\item Let $g \in G$. Then $o(g)$ divides $|G|$.
	\item For all $g \in G$ we have $g^{|G|}=e$.
	\item If $|G| = p $ where $p$ is prime then $G$ is cyclic.
	\item If $|G| < 6$ then $G$ is abelian.
	\item A \emph{left coset} is a subset of $G$ of the form $gH$.
	\item A \emph{right coset} is a subset of $G$ of the form $Hg$.
	\item If $gH=Hg$ for all $g \in G$ then we say the subgroup is normal.
	\item We denote the set of left cosets of $H$ in $G$ by $G/H$.
	\item The \emph{index} of $H \leq G$ is the number of distinct left cosets of $H$ in $G$ and $|G/H| = \frac{|G|}{|H|}.$
\end{itemize}

\textbf{Fermat's Little Theorem} If $p$ is a prime and $a \in \mathbb{Z}$ then $a^p \equiv a \mod p$.

\section{Homomorphisms and Isomorphisms}
Let $G,H$ be groups. A map $ \phi : G \rightarrow H $ is a group \emph{homomorphism} if $$ \phi(xy)=\phi(x)\phi(y) \text{ for all } x,y \in G.$$
(Product $xy$ on the left is the group operation in $G$ and the product $\phi(x)\phi(y) $ is formed using group operation in $H$.)\\
If the map is bijective then it is called an \emph{isomorphism}.

\begin{itemize}
	\item The \emph{image} of $\phi$ is im $\phi = \{h \in H | h = \phi(g) $ for some $g \in G\}$. 
	\item The \emph{kernel} of $\phi$ is ker $\phi = \{g \in G | \phi(g) = e_H\}$. 
	\item im $\phi$ is a subgroup of $H$.
	\item ker $\phi$ is a subgroup of $G$.
	\item Kernels of homomorphisms are normal subgroups.
	\item If $\phi : G \rightarrow H$ is an isomorphism then so is $\phi^{-1} : H \rightarrow G$.
	\item $\phi : G \rightarrow H$ is injective iff ker $\phi = \{e\}$.
	\item If $\phi : G \rightarrow H$ is injective then $\phi$ gives an isomorphism $G \cong $ im $\phi$.
	\item All cyclic groups of order $n$ are isomorphic, in particular every group of order 2 is isomorphic to $\mathbb{Z}_2$.
	\item Let $H,K \leq G$ with $H\cap K = \{e\}$. Then $\phi: H \times K \rightarrow HK$ given by $\phi:(h,k) \mapsto hk$ is bijective. If also $hk=kh \text{ for all } h \in H, k\in K$ then $HK$ is a subgroup of $G$ isomorphic to $H \times K$ via $\phi$.
\end{itemize}

\section{Group Actions}
Let $G$ be a group and $X$ an non empty set. Then a left action of $G$ on $X$ is a map $G \times X \rightarrow X$ such that
$$g_1 \cdot (g_2 \cdot x) = (g_1g_2)\cdot x \text{ and } e \cdot x = x $$ for all $g_1,g_2 \in G, x\in X$.
 \begin{itemize}
 	\item The \emph{kernel} of an action is the set $N = \{g \in G| g \cdot x = x \text{ for all } x\in X\}$. 
	 \item If $N=\{e\}$ (kernel is trivial) then we say the action is \emph{faithful}.
 \end{itemize}

\section{Orbit-Stabilizer}
Let $G$ act on $X$ and let $x \in X$. The \emph{stabilizer} of $x$ is $$ \text{Stab}_G (x) = \{ g \in G | g \cdot x = x\} $$ and 
the \emph{orbit} of $x$ under $G$ is \[ \text{Orb}_G (x) = \{ g \cdot x | g \in G \}. \]
\begin{itemize}
	\item The stabilizer is a subgroup of $G$.
	\item Orbits partition the set $X$.
	\item The kernel is the intersection of stabilizer subgroups, i.e. $\cap_{x \in X} \text{Stab}_G (x)$.
\end{itemize}

\columnbreak

\textbf{Orbit-Stabilizer Theorem} Let $G$ be a finite group acting on $X$, let $x\in X$. Then $$ |\text{Orb}_G (x)| \times |\text{Stab}_G (x)| = |G|.$$

\textbf{Cauchy's Theorem} If a prime $p$ divides $|G|$ then $G$ contains an element of order $p$.

\begin{itemize}
	\item An action is \emph{transitive} if for all $x,y \in X$ there exists $g\in G$ such that $y = g \cdot x\}$. Equivalently, $X$ is a single orbit under $G$.
	\item $\text{send}_x (y) = \{g \in G | g \cdot x = y \}$
	\item $\text{Fix}(g) = \{x \in X | g \cdot x = x \}$ is the \emph{fixed point set}.
	\item The number of orbits in $X = \frac{1}{|G|} \sum_{g\in G} |\text{Fix}(g)|.$
\end{itemize}

\section{Conjugacy Classes}
Let $g,h \in G$, then $h \cdot g = hgh^{-1}$ defines an action of group G on itself (\emph{conjugation action}). 
\begin{itemize}
	\item The orbits are called \emph{conjugacy classes}.
	\item We say $g_1,g_2$ are \emph{conjugate} if there exists $h \in G$ such that $g_2 =hg_1h^{-1}$, i.e. if they lie in the same conjugacy class.
	\item If $G$ is abelian then each element is its own conjugacy class.
	\item $C(g) = \{h \in G | gh=hg\}$ is the \emph{centralizer} of $g$ in $G$ and it is a subgroup of $G$.
	\item $C(G) = \{g \in G | gh=hg \text{ for all } h \in G\}$ is the \emph{centre} of a group $G$.
	\item If $g \in C(G)$ we say $g$ is \emph{central}.
	\item The centre is the intersection of all centralizers and it is a subgroup of $G$.
	\item $G$ is abelian iff $C(G) = G$.
	\item (number of conjugates of $g$ in $G$) $\times |C(g)| = |G|$.
	\item $\{e\}$ is always a conjugacy class of G.
	\item $\{g\}$ is a conjugacy class iff $g \in C(G)$. Hence $C(G)$ is the union of all one-element conjugacy classes.
	\item If $|G|=p^k$ where $p$ is prime and $k \in \mathbb{N}$, then $|C(G)| \geq p$.
\end{itemize}

Let $G$ be a group with conjugacy classes $C_1,...,C_n$ ($C_1$ is always $\{e\}$) with sizes $c_1,...,c_n$ (so $c_1 = 1$). If $g \in C_k$ then $c_k = \frac{|G|}{|C(g)|}$. In particular, $c_k$ divides the order of the group. Then the \emph{class equation of $G$} is \[|G| = c_1 +c_2 + ... +c_n.\]

	\subsection{Conjugacy in $S_n$}
	The number of elements os $S_n$ of cycle type $1^{m_1},2^{m_2},...,n^{m_n}$ is $$\frac{n!}{m_{1}!...m_{n}! 1^{m_1}2^{m_2}...n^{m_n}}.$$

\pagebreak

\section{Dihedral Group $D_n$}
We call the group of symmetries of and $n$-gon the dihedral group $D_n$.
\begin{itemize}
	\item $|D_n| = 2n$.
	\item $D_n$ is not abelian for $n \geq 3.$
\end{itemize}

\section{Symmetric Group $S_n$}
The set of all symmetries (permutations) of a set $X$ of $n$ objects is the symmetric group $S_n$.
\begin{itemize}
	\item $|S_n| = n!$.
	\item $S_n$ is abelian iff $n=2$.  
\end{itemize}

\columnbreak

\section{General Linear Group $GL(n, \mathbb{R})$}
The set of invertible $ n \times n$ matrices with entries in $\mathbb{R}$ is a group under matrix multiplication.
\begin{itemize}
	\item $GL(n, \mathbb{R})$ is not abelian. 
	\item Subgroups: \\ $SL(n, \mathbb{R})= \{A \in GL(n, \mathbb{R}) | \det{A}=1\} $, \\ $O(n, \mathbb{R})= \{A \in GL(n, \mathbb{R}) | A^T = A^{-1}\}$, \\
$SO(n, \mathbb{R})= \{A \in GL(n, \mathbb{R}) | \det{A}=1 \text{ and } A^T = A^{-1}\}$
	\item $|GL(n, \mathbb{Z}_p)| = (p^n-1)(p^n-p)(p^n-p^2)...(p^n-p^{n-1})$
\end{itemize}

\vfill\null
\columnbreak

\section{Useful facts}
\begin{itemize}
	\item If a group $G$ is cyclic then $G$ is abelian.
	\item $G$ is cyclic iff $G$ has an element of order $|G|$.
	\item If $g^2 = e \quad \forall g \in G$ then $G$ is abelian.
	\item Every group of order $p^2$ ($p$ prime) is abelian.
	\item If $H,K$ are cyclic the $H \times K$ is cyclic iff $gcd(|H|,|K|)=1$.
	\item $(gh)^{-1} = h^{-1}g^{-1}$
	\item If $G,H$ are finite subgroups that intersect trivially then $|G \times H| = |G||H|$.
	\item $o(g) = o(g^{-1})$
	\item If $G$ is abelian and $H \leq G$ then left cosets are the same as right cosets.
	\item Let $o(g) = k$ then if $k$ is even $o(g^2) = \frac{k}{2}$ and if $k$ is odd then $o(g^2) = k$.
\end{itemize}

\end{multicols}

\end{document}